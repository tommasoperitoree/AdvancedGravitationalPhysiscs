\documentclass[12pt,a4paper]{article}
\usepackage[utf8]{inputenc}
\usepackage[T1]{fontenc}
\usepackage[italian, english]{babel}

\usepackage[margin=2cm]{geometry}
\usepackage{graphicx}
\usepackage{amsmath}
\usepackage{amsfonts}
\usepackage{amssymb}
\usepackage{xcolor}
\usepackage{braket}
\usepackage{wrapfig}
\usepackage{fancyhdr}
\usepackage{hyperref}
\hypersetup{
    colorlinks,
    citecolor=black,
    filecolor=black,
    linkcolor=black,
    urlcolor=black
}

\begin{document}

\title{Advanced Gravitational Physics}
\author{Tommaso Peritore 67043A}
\date{\today}

\maketitle

\tableofcontents


\section{Gravitational Waves}

\subsection{Equations of motion}

\subsection{Solution and resonant frequency}

\subsection{System energy and radiated energy}

\subsection{\texorpdfstring{$h_\times$}{hcross} wave}

\subsection{Example and detection prospects}




\section{Cosmology}

\subsection{Equations from Section 13.3 - Dodelson\&Schmidt}

\subsubsection*{Eq.13.19 - Lensing deflection angle}

The lensing deflection angle $\Delta \boldsymbol{\theta}$ is related to the lensing potential $\phi_L$ by:
\begin{equation}
    \Delta \boldsymbol{\theta}(\mathbf{l}) = i \, \mathbf{\ell} \, \phi_L(\mathbf{l}),
\end{equation}
which is Dodelson's Eq.13.19, which we can verify starting from the real-space relation:
\begin{equation}
    \Delta \theta^i(\boldsymbol{\theta}) = \frac{\partial}{\partial \theta^i} \phi_L(\boldsymbol{\theta}),
\end{equation}
having defined the lensing potential as:
\begin{equation}
    \phi_L(\boldsymbol{\theta}) \equiv 2 \int_{0}^{\chi} \frac{d\chi'}{\chi'} \Phi(\mathbf{x}(\boldsymbol{\theta}, \chi')) (1 - \chi'/\chi).
\end{equation}


see Fourier $f(\mathbf{l}) \to e^{-i \mathbf{l} \cdot \boldsymbol{\theta}} f(\boldsymbol{\theta}) \quad ; \quad f(\boldsymbol{\theta}) \to \frac{e^{i \mathbf{l} \cdot \boldsymbol{\theta}}}{(2\pi)^2} f(\mathbf{l})$

$\phi_L(\boldsymbol{\theta}) = \int \frac{d^2l}{(2\pi)^2} \phi_L(\mathbf{l}) \, e^{i \mathbf{l} \cdot \boldsymbol{\theta}} \quad ; \quad \Delta \theta^i(\boldsymbol{\theta}) = \int \frac{d^2l}{(2\pi)^2} \Delta \theta^i(\mathbf{l}) \, e^{i \mathbf{l} \cdot \boldsymbol{\theta}}$
    
from $\Delta \theta^i(\boldsymbol{\theta}) = \frac{\partial}{\partial \theta^i} \phi_L(\boldsymbol{\theta}) \quad$ sub $\phi_L(\boldsymbol{\theta})$ w/ F

\begin{equation*}
= \frac{\partial}{\partial \theta^i} \int \frac{d^2l}{(2\pi)^2} \phi_L(\mathbf{l}) \, e^{i \mathbf{l} \cdot \boldsymbol{\theta}} = \int \frac{d^2l}{(2\pi)^2} \, i l^i \, \phi_L(\mathbf{l}) \, e^{i \mathbf{l} \cdot \boldsymbol{\theta}}
\end{equation*}

\noindent comparing \quad $\Delta \boldsymbol{\theta}(\mathbf{l}) = i \mathbf{l} \, \phi_L(\mathbf{l})$


\subsection{Exercise 13.7 - Dodelson\&Schmidt}

\subsection{CMB Polarization}


\end{document}